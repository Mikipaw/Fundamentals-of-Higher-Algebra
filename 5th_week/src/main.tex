\documentclass[12pt, a4paper]{article}

% Ru lang stuff
\usepackage [utf8x] {inputenc}
\usepackage [T2A] {fontenc}

% running titles 
\usepackage{fancybox}
\usepackage{fancyhdr}

% for last page number
\usepackage{lastpage}

%for colored tablets cells
\usepackage{colortbl}

% for Ru text in formulas
\usepackage[warn]{mathtext}

% for captions 
\usepackage[labelsep=period]{caption}
\usepackage{capt-of}

% for colored hyperrefs
\usepackage{xcolor}
\usepackage{hyperref}

% for pictures 
\usepackage{graphicx}

% for coll math
\usepackage{amsmath}
\usepackage{amsthm}

% path to all pictures
\graphicspath{{picks/}}

% for enumerates
\usepackage[shortlabels]{enumitem}

% for diff running titles on pages with diff parity
\usepackage{ifthen}
\usepackage{pdfpages}
\usepackage[strict]{changepage}

%for drawings
\usepackage{tikz}
\usetikzlibrary{calc}
\usetikzlibrary{decorations.pathmorphing}

% for good text in tablets
\usepackage{array}

% upgrading tables
\newcolumntype{P}[1]{>{\centering\arraybackslash}p{#1}}
\newcolumntype{M}[1]{>{\centering\arraybackslash}m{#1}}


% dock fields 20 15 15 35
\usepackage[left=12mm, top=12mm, right=15mm, bottom=28mm, nohead, footskip=10mm]{geometry}

% for cool tables
\usepackage{multirow}

% for different section/subsection/subsubsection styles in contents and doc
\usepackage[english, russian]{babel}

\usepackage{amsmath}

% for cool tables
\usepackage{tabularx}

\usepackage{fancyhdr,color}
\usepackage{amssymb}
% page style setup (for running titles)
\pagestyle{myheadings}
\pagestyle{fancy}
\fancyhead[C]{}
\fancyhead[R]{\rightmark}
\fancyfoot[L]{ОВАиТК, ФПМИ, МФТИ}
\fancyfoot[C]{\textcopyright Павлов М.А.}
\fancyfoot[R]{\thepage}

\fancypagestyle{plain}{ %
    \fancyhf{} % remove everything

    % lines parameters
    \renewcommand{\headrulewidth}{0pt}
    \renewcommand{\footrulewidth}{0pt}



% running titles contents
    \lfoot{\textcolor{black!50}{Павлов М. А.,}}
    \rfoot{\textcolor{black!50}{\thepage}}
}

\newcommand{\sectionmark}[1]{\markboth
{\uppercase{\thesection\hspace{1em}#1}}% левая пометка
{\uppercase{\thesection\hspace{1em}#1}}% правая пометка
}% конец макроопределения

\theoremstyle{definition}
\newtheorem{ex}{Пример}[section]
\newtheorem{st}[ex]{Утверждение}

\title{Homework 5}
\author{Mikhail Pavlov \thanks{MIPT}}
\date{March, 2022}
\begin{document}

    \section{Домашнее задание за 5-ую неделю}

    \section*{Задание 1}

    \begin{st}
        Все элементы порядка 11 сопряжены $S_{11}$
    \end{st}

        Мы знаем, что перестановки сопряжены $\Leftrightarrow$ они имеют одинаковый цикловой тип.
        НОК длин всех циклов циклового типа равен порядку перестановки.

        Поскольку 11 -- простое число, то цикловой тип тривиален: (11). Значит, все элементы порядка 11 сопряжены в $S_{11}$.

    \section*{Задание 2}

        Возьмем абелеву группу $G$ как группу $Z$ с операцией сложения. В качестве подгрупп $H_1$ и $H_2$ рассмотрим $2Z$ и $3Z$.

        Очевидно, что выбранные подгруппы изоморфны.

        Докажем, что $Z/2Z \ncong Z/3Z$.

        Нам известно, что $Z/2Z = (Z_2, +)$

        В то же время $Z/3Z = (Z_3, +)$

        Поскольку $(Z_2, +) \ncong (Z_3, +) \Rightarrow Z/2Z \ncong Z/3Z$.

    \section*{Задание 3}

        Подгруппа $H$ называется нормальной, если $\forall g \in G gH = Hg$

        Мы знаем: $g(xy)g^{-1} = (gxg^{-1})(gyg^{-1})$.

        В нашем случае: $g[x, y]g^{-1} = [gxg^{-1}, gyg^{-1}] \Rightarrow g([x_1, y_1]\dots[x_n, y_n])g^{-1} = (g[x_1, y_1]g^{-1})\dots(g[x_n, y_n]g^{-1}) = [gx_1g^{-1}, gy_1g^{-1}]\dots[gx_ng^{-1}, gy_ng^{-1}]$

        То есть $\forall g \in G gHg^{-1} \in H \Rightarrow H \lhd G$.

    \section*{Задание 4}

        Рассмотрим автоморфизм $\phi_g : h \rightarrow ghg^{-1}$ (он действительно является автоморфизмом, т.к. есть взаимно однозначное соответсвие и $\phi(h_1 h_2) = \phi(h_1) \cdot \phi (h_2)$).

        В таком случае мы получаем, что рассматриваемая группа изоморфна подгруппе $H$, откуда следует, что она является подгруппой группы $G$.

    \end{document}

