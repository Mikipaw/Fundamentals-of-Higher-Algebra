\documentclass[12pt, a4paper]{article}

% Ru lang stuff
\usepackage [utf8x] {inputenc}
\usepackage [T2A] {fontenc}

% running titles 
\usepackage{fancybox}
\usepackage{fancyhdr}

% for last page number
\usepackage{lastpage}

%for colored tablets cells
\usepackage{colortbl}

% for Ru text in formulas
\usepackage[warn]{mathtext}

% for captions 
\usepackage[labelsep=period]{caption}
\usepackage{capt-of}

% for colored hyperrefs
\usepackage{xcolor}
\usepackage{hyperref}

% for pictures 
\usepackage{graphicx}

% for coll math
\usepackage{amsmath}
\usepackage{amsthm}

% path to all pictures
\graphicspath{{picks/}}

% for enumerates
\usepackage[shortlabels]{enumitem}

% for diff running titles on pages with diff parity
\usepackage{ifthen}
\usepackage{pdfpages}
\usepackage[strict]{changepage}

%for drawings
\usepackage{tikz}
\usetikzlibrary{calc}
\usetikzlibrary{decorations.pathmorphing}

% for good text in tablets
\usepackage{array}

% upgrading tables
\newcolumntype{P}[1]{>{\centering\arraybackslash}p{#1}}
\newcolumntype{M}[1]{>{\centering\arraybackslash}m{#1}}


% dock fields 20 15 15 35
\usepackage[left=12mm, top=12mm, right=15mm, bottom=28mm, nohead, footskip=10mm]{geometry}

% for cool tables
\usepackage{multirow}

% for different section/subsection/subsubsection styles in contents and doc
\usepackage[english, russian]{babel}

\usepackage{amsmath}

% for cool tables
\usepackage{tabularx}

\usepackage{fancyhdr,color}
\usepackage{amssymb}
% page style setup (for running titles)
\pagestyle{myheadings}
\pagestyle{fancy}
\fancyhead[C]{}
\fancyhead[R]{\rightmark}
\fancyfoot[L]{ОВАиТК, ФПМИ, МФТИ}
\fancyfoot[C]{\textcopyright Павлов М.А.}
\fancyfoot[R]{\thepage}

\fancypagestyle{plain}{ %
    \fancyhf{} % remove everything

    % lines parameters
    \renewcommand{\headrulewidth}{0pt}
    \renewcommand{\footrulewidth}{0pt}



% running titles contents
    \lfoot{\textcolor{black!50}{Павлов М. А.,}}
    \rfoot{\textcolor{black!50}{\thepage}}
}

\newcommand{\sectionmark}[1]{\markboth
{\uppercase{\thesection\hspace{1em}#1}}% левая пометка
{\uppercase{\thesection\hspace{1em}#1}}% правая пометка
}% конец макроопределения

\theoremstyle{definition}
\newtheorem{ex}{Пример}[section]
\newtheorem{st}[ex]{Утверждение}

\title{Homework 7}
\author{Mikhail Pavlov \thanks{MIPT}}
\date{March, 2022}
\begin{document}

    \section{Домашнее задание за 7-ую неделю}

    \section*{Задание 1}

        Приведем доказательство пункта 2, т.к. из него следует и первый:

        2) Следует доказать, что группа порядка 15 является абелевой.

        Докажем, что центр группы не тривиален (от противного):

        Пусть он тривиален. Тогда $\exists x, y: 3x + 5y + 1 = 15 \Rightarrow x = 3, y = 1$ -- единственное решение из натуральных чисел.
        Получается, у нас есть 3 орбиты из трех элементов и одна из пяти. То есть 5 элементов группы коммутирует с тремя элементами порядка 3.

        Однако, если какой-то элемент $g$ коммутирует с тремя элементами, то и $g^2$ коммутирует с тремя элементами, откуда следует четность количества таких элементов.

        Значит, центр группы нетривиален и элемент порядка 5 коммутирует с каким-то элементом порядка 3. Их произведение равно 15, из чего следует, что группа циклическая.
        Кроме того, подгруппа $H$ порядка 5 лежит в центре. Но поскольку факторгруппа неабелевой группы по центру не может быть циклической, то мы получаем противоречие.

        Тогда получается, что центр не совпадает с $H$, откуда следует, что он равен самой группе и в таком случае группа $G$ абелева.

    \section*{Задание 2}

        Нам необходимо найти различные раскраски $f: \{a, b, c, d\} \rightarrow \{1, 2\}$

        Квадрат мы можем повернуть на $0, \frac{\phi}{2}, \phi$ и $\frac{3 \phi}{2}$.

    \begin{center}
        \begin{tabular}{ | c | c | }
            \hline
            $S_4$ & кол-во  \\ \hline
            () & 1  \\ \hline
            (2) & 2  \\ \hline
            (2)(2) & 1  \\ \hline
            (4) & 1  \\ \hline
        \end{tabular}
    \end{center}

    Всего возможных раскрасок -- $2^4 = 16$

    Порядок группы квадрата $|G| = 4$.

    Тогда по Лемме Бернсайда количество раскрасок $r = \frac{2^4 \cdot 1 + 1 \cdot 2^1 + 2^1 \cdot 2 + 2^1 \cdot 1}{4} = \frac{24}{4} = 6$

    Ответ: 6 раскрасок


    \end{document}

