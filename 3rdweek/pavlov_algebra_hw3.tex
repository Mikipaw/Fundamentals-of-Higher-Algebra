\documentclass[12pt, a4paper]{article}

% Ru lang stuff
\usepackage [utf8x] {inputenc}
\usepackage [T2A] {fontenc}

% running titles 
\usepackage{fancybox}
\usepackage{fancyhdr}

% for last page number
\usepackage{lastpage}

%for colored tablets cells
\usepackage{colortbl}

% for Ru text in formulas
\usepackage[warn]{mathtext}

% for captions 
\usepackage[labelsep=period]{caption}
\usepackage{capt-of}

% for colored hyperrefs
\usepackage{xcolor}
\usepackage{hyperref}

% for pictures 
\usepackage{graphicx}

% for coll math
\usepackage{amsmath}
\usepackage{amsthm}

% path to all pictures
\graphicspath{{picks/}}

% for enumerates
\usepackage[shortlabels]{enumitem}

% for diff running titles on pages with diff parity
\usepackage{ifthen}
\usepackage{pdfpages}
\usepackage[strict]{changepage}

%for drawings
\usepackage{tikz}
\usetikzlibrary{calc}
\usetikzlibrary{decorations.pathmorphing}

% for good text in tablets
\usepackage{array}

% upgrading tables
\newcolumntype{P}[1]{>{\centering\arraybackslash}p{#1}}
\newcolumntype{M}[1]{>{\centering\arraybackslash}m{#1}}


% dock fields 20 15 15 35
\usepackage[left=12mm, top=12mm, right=15mm, bottom=28mm, nohead, footskip=10mm]{geometry}

% for cool tables
\usepackage{multirow}

% for different section/subsection/subsubsection styles in contents and doc
\usepackage[english, russian]{babel}

\usepackage{amsmath}

% for cool tables
\usepackage{tabularx}

\usepackage{fancyhdr,color}
% page style setup (for running titles)
\fancypagestyle{plain}{ %
\fancyhf{} % remove everything

 % lines parameters
\renewcommand{\headrulewidth}{0pt}
\renewcommand{\footrulewidth}{0pt}

% running titles contents
\lfoot{\textcolor{black!50}{Павлов М. А.,}}
\rfoot{\textcolor{black!50}{\thepage}}
}

\theoremstyle{definition}
\newtheorem{ex}{Пример}[section]
\newtheorem{st}[ex]{Утверждение}

\title{Homework 2}
\author{Mikhail Pavlov \thanks{MIPT}}
\date{Febrary, 2022}
\begin{document}

\section{Task}

$\Leftarrow$

\begin{proof}
    $t \cdot k \equiv 1 mod n \Leftrightarrow kt = 1 + qn$

    $k = k' \cdot d$

    $n = n' \cdot d$

    $d$ -- НОД($k, n$)

    Тогда из $kt = 1 + qn$ получаем $1 = k'dt - qn'd = (k't - qn')d$

    В таком случае $d | 1 \Rightarrow d = 1$
\end{proof}

$\Rightarrow$

\begin{proof}
    Пусть $A$ -- множество чисел, которые можно получить из $k, n$ с помощью сложения и вычитания.

    Тогда $r_1 = k - nq_1, r_2 = n - r_1q_2, r_3 = r_1 - r_2q_3, ...$

    $r_1 \in A \Rightarrow r_2 \in A \Rightarrow r_3 \in A \Rightarrow ... \Rightarrow r_n \in A$.

    А $r_n$ как раз равно $d$.

    Так как $k$ и $n$ взаимно просты, то $d = 1 = k \widetilde{k} + n \widetilde{n} \Rightarrow k \widetilde{k} = 1 (mod n) \Rightarrow \exists t : kt = 1 (mod n)$.

\end{proof}

\section{Task}

    $17^{668} \equiv (-10)^{668} \equiv 10^{668} (mod 27)$.

    $10^{668} = 1 + 9...9$ (668 девяток) -- делится на 9, но не делится на 27 (т.к. при делении на 9 остается 668 единиц, где сумма цифр не делится на 3).

    Если вычесть 9 и оставить 667 девяток, то число также не будет делиться на 27. Значит, необходимо вычесть 18.

    Таким образом, мы получим остаток от деления 1 + 18 = \textbf{19}.

\section{Task}

    Методом пристального взгляда заметим, что $2^{3n} \equiv 8 (mod 14)$. (То есть у нас тут получается циклическая группа порядка 3 остатков от деления на 14 чисел вида $2^n$).

    Заметим также, что $2^{21^{42069}} = 2^{3n}$, т.к. $21 \vdots 3$.
    
    Таким образом, мы приходим к ответу \textbf{8}.

\section{Task}

    \subsection{Part}

        Сначала проверим на гомоморфизм: $\phi (a^{k_1}, a^{k_2}) \cdot \phi (a^{m_1}, a^{m_2}) =\rightarrow \phi (a^{m_1 * k_1}, a^{m_2 * k_2})$ -- гомоморфизм.

        Этот гомоморфизм биективен, т.к. каждому элементу $(a^k, a^m)$ можно поставить в соответствие $a^(11k + m)$.

        Значит, данные группы изоморфны.

    \subsection{Part}

        Данные группы не гомоморфны, т.к. биективное отображение из первой группы во вторую не сохраняет групповую операцию, а именно
        возникают проблемы с элементами вида $(a^6, a^16), (a^4, a^17)$ (изоморфизм сохраняет порядки элементов, а здесь этого, к сожалению, не наблюдается).

        То есть данные группы \textbf{не изоморфны}


\section{Task}

    Пусть $\phi$ -- автоморфизм нашей группы.
    Если $\phi (1) = d \in Z$, то $\forall z \in Z \phi (z) = \phi (1 + ... + 1) = \phi (1) + ... + \phi (1) = dz$.

    Таким образом, все автоморфизмы вида $\phi : G_1 \rightarrow G_2 : \phi (x) = dx$ нам подходят.

\section{Task}

    Предположим, что $\phi (x_1) = \phi (x_2)$, т.е. $ax_1 = ax_2$.

    Введем $t = x_1 - x_2$. Тогда $at = 0$. Значит, $|z| | a$. По теореме Лагранжа порядок $z$ также делит порядок группы $G$.

    Т.к. $a$ и $|G|$ взаимно просты, то $|t| = 1 \Rightarrow z = 1$ (нейтральный элемент), т.к. только нейтральный элемент может быть порядка 1.
    В таком случае мы получаем $x_1 = x_2$. Значит, отображение $\phi$ инъективно.
    
    Т.к. $\phi$ -- инъективное отображение множества в себя, то оно, очевидно, биективно.


\section{Task}

    Т.к. $|G|$ = НОК длин непересекающихся циклов, то если НОК каких-то чисел равно 3, то среди этих чисел могут быть только 1 и 3.
    
    Т.к. циклы длины 3 есть четные перестановки, то любое их произведение также будет четной перестановкой, откуда следует, что перестановки порядка 3 \textbf{не порождают} группу $S_{33}$.

\section{Task}

    Стоит отметить, что для доказательства утверждения нам достаточно доказать, что $\forall a, b \in G_1 f(b)f(a) = f(a)f(b)$.

    Итак, мы имеем $f(ab) = f(b)f(a)$ и $f(ba) = f(a)f(b)$.

    $f(ab)f^{-1}(a) = f(b)f(a)f^{-1}(a)$

    $f(ab)f^{-1}(a) = f(b)$

    $f(ab)f(a^{-1}) = f(b)$

    $f(a)f(ab)f^{-1}(a) = f(a)f(b) = f(ba)$

    $f(ab) = f(ba) \Rightarrow$ данные группы изоморфны. 

\section{Task}

    Если $m, n \in H(G) \Rightarrow mn \in H(G)$
    
    $P_{mn} = P_m P_n$

    $P_n = P^{-1}_m P_{mn}$

    Обозначим $q = mn$:

    Если $m \in A(G), q \in H(G)$ и $m | q$, то $\frac{q}{m} \in H(G)$.

    $x^{-n}y^{-n} = (yx)^{-n} \Rightarrow x^{1-n}y^{1-n} = (xy)^{1-n}.$

    Если $n \in H(G) \Rightarrow 1 - n \in H(G)$

    Аналогично доказываем, что $n - 1 \in H(G)$

    Т.к. $1 - n \in H(G)$ и $n - 1 \in H(G)$, то группа G -- абелева.

\section{Task}

    \textbf{Да, могут.}

    В качестве примера возьмем $A$ -- группу из множества рациональных многочленов с операцией сложения, а $B$ -- подгруппа,
    у которой свободный член -- целое число. Тогда группа с множеством многочленов без свободного члена будет будет изоморфна $A$.
    Однако, в таком случае для элемента $1 \in B$ не будет элемента $b' \in B : 2b' = 1$. Значит, $A$ и $B$ неизоморфны.
    

\end{document}