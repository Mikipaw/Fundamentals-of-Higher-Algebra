\documentclass[12pt, a4paper]{article}

% Ru lang stuff
\usepackage [utf8x] {inputenc}
\usepackage [T2A] {fontenc}

% running titles 
\usepackage{fancybox}
\usepackage{fancyhdr}

% for last page number
\usepackage{lastpage}

%for colored tablets cells
\usepackage{colortbl}

% for Ru text in formulas
\usepackage[warn]{mathtext}

% for captions 
\usepackage[labelsep=period]{caption}
\usepackage{capt-of}

% for colored hyperrefs
\usepackage{xcolor}
\usepackage{hyperref}

% for pictures 
\usepackage{graphicx}

% for coll math
\usepackage{amsmath}
\usepackage{amsthm}

% path to all pictures
\graphicspath{{picks/}}

% for enumerates
\usepackage[shortlabels]{enumitem}

% for diff running titles on pages with diff parity
\usepackage{ifthen}
\usepackage{pdfpages}
\usepackage[strict]{changepage}

%for drawings
\usepackage{tikz}
\usetikzlibrary{calc}
\usetikzlibrary{decorations.pathmorphing}

% for good text in tablets
\usepackage{array}

% upgrading tables
\newcolumntype{P}[1]{>{\centering\arraybackslash}p{#1}}
\newcolumntype{M}[1]{>{\centering\arraybackslash}m{#1}}


% dock fields 20 15 15 35
\usepackage[left=12mm, top=12mm, right=15mm, bottom=28mm, nohead, footskip=10mm]{geometry}

% for cool tables
\usepackage{multirow}

% for different section/subsection/subsubsection styles in contents and doc
\usepackage[english, russian]{babel}

\usepackage{amsmath}

% for cool tables
\usepackage{tabularx}

\title{Homework 1}
\author{Mikhail Pavlov \thanks{MIPT}}
\date{Febrary, 2022}
\begin{document}

\section*{Task 1}

\begin{proof}
    Рассмотрим такие числа $a$ и $b$, что $b = a^{-1}$ и $b \cdot a = e$. (такое число $b$ найдется для любого $a$ по условию)

    Аналогично запишем для $b$, т.е. найдется $c = b^{-1} : c \cdot b = e$. 

    Далее получаем:

    \begin{equation*}
        \begin{cases}
            c \cdot (b \cdot a) = c \cdot e \\
            (c \cdot b) \cdot a = e \cdot a    
        \end{cases}
    \end{equation*}
    
    По 1-ому свойству моноида $c \cdot (b \cdot a) = (c \cdot b) \cdot a$, откуда следует, что $c \cdot e = e \cdot a$.

    Далее, воспользовавшись 2-ым свойством моноида, получаем $c = a$, откуда $a \cdot b = b \cdot a = e$, где $b = a^{-1}$,
    что является 3-им свойством группы.
\end{proof}

\section*{Task 2}

Таблица Кэли для группы G порядка 4:

\begin{center}
    \begin{tabular}{ | c | c | c | c | c | }
        \hline
        . & e & a & b & c \\ \hline
        e & e & a & b & c \\ \hline
        a & a & e & c & b \\ \hline
        b & b & c & e & a \\ \hline
        c & c & b & a & e \\ \hline
    \end{tabular}
\end{center} 

Явный пример группы: (Z/4Z, |-|), где бинарный оператор |-| работает следующим образом: x |-| y = |x - y| при x + y $\neq$ 3 и |x + y| при x + y = 3. 

\begin{center}
    \begin{tabular}{ | c | c | c | c | c | }
        \hline
        . & 0 & 1 & 2 & 3 \\ \hline
        0 & 0 & 1 & 2 & 3 \\ \hline
        1 & 1 & 0 & 3 & 2 \\ \hline
        2 & 2 & 3 & 0 & 1 \\ \hline
        3 & 3 & 2 & 1 & 0 \\ \hline
    \end{tabular}
\end{center} 

\section*{Task 3}

    Ответ: нет, не образует, т.к. не выполнено 2-ое свойство группы.

\section*{Task 4}

\begin{proof}
    
    1) Начнем с простого: из курса линейной алгебры нам известно, что произведение матриц ассоциативно,
    значит, первое свойство выполняется.

    2) Второе свойство тоже очевидно и следует непосредственно из существования единичной матрицы размера $n$.

    3) Сначала стоит вспомнить тот факт, что произведение верхней треугольной матрицы на другую верхнюю треугольную матрицу дает в результате верхнюю треугольную матрицу
    (в этом можно убедиться, если расписать матричное произведение: когда мы будем считать значения в ячейках нижнего треугольника, у нас в каждом слагаемом либо элемент из 1-ой матрицы будет нулевым, либо элемент из второй матрицы.
    В результате получится сумма из $n$ нулей, которая, очевидно, в итоге даст 0).
    Также стоит отметить, что обратная матрица всегда будет существовать, т.к. определитель любой матрицы из нашей группы ненулевой (равен 1).
    Значит, 3-е свойство тоже выполнено.

    Поскольку все три свойства выполняются, то рассматриваемое множество образует группу по операции матричного умножения.
\end{proof}     

\end{document}