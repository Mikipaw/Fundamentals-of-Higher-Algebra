\documentclass[12pt, a4paper]{article}

% Ru lang stuff
\usepackage [utf8x] {inputenc}
\usepackage [T2A] {fontenc}

% running titles 
\usepackage{fancybox}
\usepackage{fancyhdr}

% for last page number
\usepackage{lastpage}

%for colored tablets cells
\usepackage{colortbl}

% for Ru text in formulas
\usepackage[warn]{mathtext}

% for captions 
\usepackage[labelsep=period]{caption}
\usepackage{capt-of}

% for colored hyperrefs
\usepackage{xcolor}
\usepackage{hyperref}

% for pictures 
\usepackage{graphicx}

% for coll math
\usepackage{amsmath}
\usepackage{amsthm}

% path to all pictures
\graphicspath{{picks/}}

% for enumerates
\usepackage[shortlabels]{enumitem}

% for diff running titles on pages with diff parity
\usepackage{ifthen}
\usepackage{pdfpages}
\usepackage[strict]{changepage}

%for drawings
\usepackage{tikz}
\usetikzlibrary{calc}
\usetikzlibrary{decorations.pathmorphing}

% for good text in tablets
\usepackage{array}

% upgrading tables
\newcolumntype{P}[1]{>{\centering\arraybackslash}p{#1}}
\newcolumntype{M}[1]{>{\centering\arraybackslash}m{#1}}


% dock fields 20 15 15 35
\usepackage[left=12mm, top=12mm, right=15mm, bottom=28mm, nohead, footskip=10mm]{geometry}

% for cool tables
\usepackage{multirow}

% for different section/subsection/subsubsection styles in contents and doc
\usepackage[english, russian]{babel}

\usepackage{amsmath}

% for cool tables
\usepackage{tabularx}

\usepackage{fancyhdr,color}
% page style setup (for running titles)
\fancypagestyle{plain}{ %
\fancyhf{} % remove everything

 % lines parameters
\renewcommand{\headrulewidth}{0pt}
\renewcommand{\footrulewidth}{0pt}

% running titles contents
\lfoot{\textcolor{black!50}{Гущин Д. Д., Павлов М. А., МФТИ, ФПМИ}}
\rfoot{\textcolor{black!50}{\thepage}}
}

\theoremstyle{definition}
\newtheorem{ex}{Пример}[section]
\newtheorem{st}[ex]{Утверждение}

\title{Homework 2}
\author{Mikhail Pavlov \thanks{MIPT}}
\date{Febrary, 2022}
\begin{document}

\section{}

\begin{st}
    В циклической группе конечного порядка всякая подгруппа является циклической.
\end{st}

\begin{proof}
   
    Рассмотрим группу $G$ : $a$ -- порождающий элемент, и подгруппу $H < G$ : $\exists k \text{(минимальное)} : a^k \in H$.
    
    Предположим, что в $H$ найдется также элемент $a^l$, где $l > k$ и $l$ не делится нацело на $k$.
    
    По аксиоме 3 группы $\exists a^{-k}$ и из замкнутости $a^{-k} \cdot a^{l} \in H$. В таком случае $a^{l (mod k)} \in H$, но $l (mod k) < k$, то есть мы пришли к противоречию, откуда следует, что $a^l$ не лежит в $H$.
    
    Таким образом, $a^k$ -- порождающий элемент циклической подгруппы $H$.

\end{proof}

\section{}

Пусть $a_0$ -- порождающий элемент. Тогда $a = a_0^d$.

В таком случае $a^k = (a_0^d)^k = a_0^{d \cdot k}$.

Значит, порядок рассматриваемого нами элемента равен \textbf{$d \cdot k$}.

\section{}

Из теоремы Лагранжа следует, что подгруппами циклической группы $G$ порядка $n$ будут такие группы $H_1, ..., H_i, ...$, что $i | n$, т.е. все группы с порождающими элементами, равными целочисленным делителям числа $n$. 
  

\section{}

\subsection{}

Пусть $a$ -- порождающий элемент нашей группы $C_m$, а $a^q$ -- решение уравнения $x^k = e$ в данной группе.

Обозначим  $d = $ НОД($m, k$).

Тогда $(a^q)^k = e \Rightarrow m | kq \Rightarrow n | qd$.

В итоге мы получаем $q = \frac{m}{d} \cdot l$, где $l \in \{0, 1,..., d - 1\}$.

То есть количество решений равно $|\{0, 1,..., d - 1\}| = d$. 

\subsection{}

Мы имеем $(a^t)^10 = e$, где $a$ -- порождающий элемент группы $C_{100}$.

$t = \frac{100}{10} \cdot i = 10 \cdot i,$ где $ i = \{0, 1,..., 9\}$. 

\section{}

\begin{proof}

    Как мы выяснили в \textbf{Task 4}, количество решений уравнения в циклической группе равно НОД($m, k$). В нашем случае $m = 12, k$ -- какое-то число, очевидно, большее 12.

    Какое бы число $k$ мы ни взяли, НОД($12, k) \leq 12 < 14$, значит, группа $G$ не является циклической.

\end{proof}

\section{}

\subsection{}

Проверим свойства подгруппы для $C(G)$:

1) Верно, т.к. из аксиомы группы $\forall g \in G ge = eg$.

2) Если $a, b \in C(G)$, то $\forall g g(ab) = gab = agb = (ab)g \Rightarrow ab \in C(G)$.  

3) $gx = xg \Rightarrow x^{-1}gx = x^{-1}xg = g \Rightarrow x^{-1}gxx^{-1} = gx^{-1} \Rightarrow x^{-1}g = gx^{-1} \Rightarrow x^{-1} \in C(G)$.

Все свойства выполнены, значит, $C(G)$ -- подгруппа $G$.

\subsection{}

Аналогично проверяем свойства:

1) Верно (аналогично док-ву в пред. половине задания)

2) Если $g_1, g_2 \in N(S)$, то $S(g_1g_2) = g_1Sg_2 = (g_1g_2)S$.

3) $Sg = gS \Rightarrow S = Sgg^{-1} = gSg^{-1} \Rightarrow g^{-1}S = g^{-1}gSg^{-1} = Sg^{-1}$.

Все свойства выполнены, значит, $N(S)$ -- подгруппа $G$.

\end{document}